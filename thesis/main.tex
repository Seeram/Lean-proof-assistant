\documentclass[a4paper,10pt,oneside]{report}%pridat twoside, do [] pre obojstrannu tlac
\pagestyle{headings}
\usepackage[top=2.5cm, bottom=2.5cm, left=3.5cm, right=2cm]{geometry} %odporucane okraje
\linespread{1.50}

%% Generally used
\usepackage{amsthm}
\usepackage{amsmath}
\usepackage{amssymb}
\usepackage{mathtools}
\usepackage{listings}
%% Generally used

%% Covering
\usepackage{pict2e,picture}

\newcommand{\coveringA}{%
  \mathrel{-\mkern-4mu}<%
}
\newcommand{\coveringB}{\mathrel{\text{$\vcenter{\hbox{\pictcoveringB}}$}}}

\newcommand{\pictcoveringB}{%
  \begin{picture}(1em,.5em)
  \roundcap
  \put(0,.25em){\line(1,0){.6em}}
  \put(.6em,.25em){\line(3,1){.4em}}
  \put(.6em,.25em){\line(3,-1){.4em}}
  \end{picture}%
}
%% Covering

%% Powerset
\newcommand{\powerset}{\raisebox{.15\baselineskip}{\Large\ensuremath{\wp}}}
%% Powerset

\newtheorem{theorem}{Theorem}

\author{Mat\'u\v{s} Behun}
\title{Lattice theory notes}

\begin{document}

\section{Úvod}

Pri procese rozširovania matematickej teórie vytvárame tvrdenia generalizujúce jej princípy.
Ak chceme aby náša teória bola správna, všetky jej tvrdenia musia byť logicky odvodené z postulátov alebo tvrdení z nich odvodených.
Potvrdenie správnosti tvrdenia, vyslovením predpokladu, axiómu alebo napísaním formule ktorú dostaneme aplikáciou dedukčného pravidla na niektoré v postupnosti predchádzajúce formule nazývame dôkazom.

% Je to pravda?
Z kvalitatívneho hľadiska pri vyslovení dôkazu uvažujeme o všeobecnosti dôkazu a správnosti aplikácie dedukčného pravidla.
% Je to vhodný príklad(chcel som uviest priklad na prvy pohlad spravneho tvrdenia ktore bolo dokazane ako nespravne ? Ako správne citovať? Nápad som mal odtiaľto https://en.wikipedia.org/wiki/List_of_disproved_mathematical_ideas
O nutnosti korektného dokazovania tvrdení hovorí napríklad tvrdenie z teórie čísel o hornom ohraničení počtu prvočísel logaritmickým integrálom.

\begin{equation}
    \pi(x) \leq \int_{0}^{x} \frac{1}{ \ln{t} } dt
\end{equation}

Tvrdenie bolo považované za správne Bernhardom Riemannom a evidencia to taktiež naznačovala.
Neskôr sa ukázalo že tvrdenie nie je správne pri čísle pod hodnotou $10^{317}$.
% Doplniť rok
Veta o 4 farbách ktorá bola vyslovená v roku 1852 Francisom Guthrie ktorá hovorí, že každá rovinná mapa je zafarbiteľná 4 farbami.
% 18-faces
Táto veta bola nesprávne dokázaná v roku Kempom (1879) and Taitom (1880). Kempov dôkaz bol vyvrátený o 10 rokov mapov s 18 stenami.
% Je lepšie skloňovať cudzie mená?
Pri dôkaze tejto vety bol neskôr v roku 1977 Appelom and Hakenom z časti využitý počítač pre kontrolu špeciálnych diskrétnych prípadov.

\section{Počítačom asistované dokazovanie}

Specializacia 
Typy softverov a na akych principoch su zalozene, napr. programy pre asistovane dokazovanie je zalozena na dependent type theory


\subsection{Výroková logika}

V prípade že chceme aby výrokove formuly korenšpondovali s typmi.
Ich booleova reprezentácia s hodnotami $1,0$ je nahradená otázkou existencie prvkov v množine.
V prípade implikácie o existencii funkcie v množine.
Funkcie v programoch ale môžu mať pri rovnakých vstupoch a výstupoch mať rôznu výpočtovú zložitosť.
Dôvod prečo by sme sa mali pozerať na dôkazy(podľa publikácie Gir11) v troch rovinách.

1. Booleovský - tvrdenia sú booleovské hodnoty, zaujímame sa o dokázateľnosť tvrdenia
2. Existenčný - tvrdenia sú množiny, aké funkcie môžu byť
3. Úmyselný - zaujímame sa o zložitosť vytvoreného dôkazu a ako sa zjednoduší cez (cut eliminitation)

\subsubsection{Intuicionizmus}

Tento posunu od eixstencie dôkazu k dokázateľnosti začal z filozfie Brouwer s počiatkom v 20. storočí sa nazývy intuitionizmus.

Z intuionistického pohľadu by mali byť premennné výrokových formúl interpretované ich dôkazy. Interpretácia formúl sa potom zmení

$ A \wedge B $ ako $ A \times B $

$ A \vee B $ ako $ A \sqcup B $ zjednotenie rozdielu

$ A \implies B $ spôsob skonštruovania dôkazu $ B $ z dôkazu $ A $

$ \neg A = A \implies \perp $ existencie kontrapríkladu

Z tohto pohľadu bolo Brouwerom odmietnutý princíp ktorý platí v klasickej logike $ \neg \neg A $ 

[Gir11] Jean-Yves Girard.The Blind Spot: lectures on logic. EuropeanMathematical Society, 2011
% sekcia 7.1

\subsubsection{Formalizovanie dôkazu}

Dôkaz z teórie usporiadania. Tak ako je Program = Proof

Otázka ohľadom konzistentnosti dôkazu.

Otázka kontrola typov by mala byť rozhodnuteľná.

Definicia formuly, postupnosti, kontextu, fragmentov ktoré navazuju na typovy lambda calculus

\subsection{Typový lambda calculus}

\subsection{Predikátová logika}

\subsection{Teória zavislostných typov}

\section{Lean-theorem-prover}
\subsection{Constracting proof}
\subsection{Forward proofs}
    \subsubsection{assume}
    \subsubsection{calc}
    \subsubsection{fix}
    \subsubsection{have}
    \subsubsection{let}
    \subsubsection{show}
\subsection{Backward proofs}
    \subsubsection{cc}
    \subsubsection{clear}
    \subsubsection{exact}
    \subsubsection{induction}
    \subsubsection{intro}
    \subsubsection{refl}
    \subsubsection{refl}
\subsubsection{Inductive types}

% sposob ukladania kodu
\begin{verbatim}
    structure point :=
      ( x : nat )
      ( y : nat )

    /-- alternative notation -/
    structure point_alternative :=
      mk :: (x : nat) (Y : nat)

    def p1 : point :=
    {
      x   := 10,
      y   := 20,
    }

    /- same point, different notation, same notation for ordered seti -/
    def p2 : point := $\langle 10, 20 \rangle$

    /- instance only one part of structure, rest implicitly from other instance
    def p3 : point := {
        x := 20,
        ..p
    }
\end{verbatim}

\subsubsection{Type classes}




\end{document}

\begin{theorem} \emph{Veta o izomorfizme modulárnych zväzov}
Nech L je modulárnym zväzom a $a, b \in L$. Potom
    \begin{equation}
        \varphi_{b}: x \mapsto x \wedge b, x \in [a, a \vee b],
    \end{equation}
Je izomorfizmom medzi intervalmi $[a, a \vee b]$ a $[ a \wedge b, b]$.
Inverzným izomorfizmom je
    \begin{equation}
        \psi_{a}: y \mapsto x \vee a, y \in [a \wedge b, b].
    \end{equation}
\end{theorem}
\emph{Dôkaz}.
Stačí ukázať že $\varphi_{b}\psi_{a}(y) = y$ pre všetky $x \in [a, a \vee b]$.
Z duality vyplýva že $\varphi_{b}\psi_{a}(y) = y$ pre všetky
    $y \in [a \wedge b, b]$,
Majme $x \in [a, a \vee b]$. Potom
    $\psi_{a}\varphi_{b} = ( x \wedge b ) \vee a$ nerovnosť $a \leq x$ platí
    potom aj modularita
    \begin{equation}
        \varphi_{a}\psi_{b}(x) =
        ( x \wedge b ) \vee a =
        x \wedge ( b \vee a) =
        x
    \end{equation}
    pretože
    \[
        \pushQED{\qed}
        x \leq a \vee b. \qedhere
        \popQED
    \]
