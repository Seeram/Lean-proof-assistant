\documentclass[a4paper,12pt,oneside]{report}%pridat twoside, do [] pre obojstrannu tlac
\pagestyle{headings}
\usepackage[top=2.5cm, bottom=2.5cm, left=3.5cm, right=2cm]{geometry} %odporucane okraje
\linespread{1.50}

%% Generally used
\usepackage{amsthm}
\usepackage{amsmath}
\usepackage{amssymb}
\usepackage{mathtools}
%% Generally used

%% Covering
\usepackage{pict2e,picture}

\newcommand{\coveringA}{%
  \mathrel{-\mkern-4mu}<%
}
\newcommand{\coveringB}{\mathrel{\text{$\vcenter{\hbox{\pictcoveringB}}$}}}

\newcommand{\pictcoveringB}{%
  \begin{picture}(1em,.5em)
  \roundcap
  \put(0,.25em){\line(1,0){.6em}}
  \put(.6em,.25em){\line(3,1){.4em}}
  \put(.6em,.25em){\line(3,-1){.4em}}
  \end{picture}%
}
%% Covering

%% Powerset
\newcommand{\powerset}{\raisebox{.15\baselineskip}{\Large\ensuremath{\wp}}}
%% Powerset

\newtheorem{theorem}{Theorem}

\author{Mat\'u\v{s} Behun}
\title{Lattice theory notes}

\begin{document}

\section{Chapter 1}

\begin{theorem} \emph{Veta o izomorfizme modulárnych zväzov}
Nech L je modulárnym zväzom a $a, b \in L$. Potom
    \begin{equation}
        \varphi_{b}: x \mapsto x \wedge b, x \in [a, a \vee b],
    \end{equation}
Je izomorfizmom medzi intervalmi $[a, a \vee b]$ a $[ a \wedge b, b]$.
Inverzným izomorfizmom je
    \begin{equation}
        \psi_{a}: y \mapsto x \vee a, y \in [a \wedge b, b].
    \end{equation}
\end{theorem}
\emph{Dôkaz}.
Stačí ukázať že $\varphi_{b}\psi_{a}(y) = y$ pre všetky $x \in [a, a \vee b]$.
Z duality vyplýva že $\varphi_{b}\psi_{a}(y) = y$ pre všetky
    $y \in [a \wedge b, b]$,
Majme $x \in [a, a \vee b]$. Potom
    $\psi_{a}\varphi_{b} = ( x \wedge b ) \vee a$ nerovnosť $a \leq x$ platí
    potom aj modularita
    \begin{equation}
        \varphi_{a}\psi_{b}(x) =
        ( x \wedge b ) \vee a =
        x \wedge ( b \vee a) =
        x
    \end{equation}
    pretože
    \[
        \pushQED{\qed}
        x \leq a \vee b. \qedhere
        \popQED
    \]
\end{document}
