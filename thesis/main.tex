\documentclass[a4paper,10pt,oneside]{report}%pridat twoside, do [] pre obojstrannu tlac
\pagestyle{headings}
\usepackage[top=2.5cm, bottom=2.5cm, left=3.5cm, right=2cm]{geometry} %odporucane okraje
\linespread{1.50}

%% Generally used
\usepackage{ebproof}
\usepackage{amsthm}
\usepackage{amsmath}
\usepackage{amssymb}
\usepackage{mathtools}
\usepackage{listings}
%% Generally used

%% Covering
\usepackage{pict2e,picture}

\newcommand{\coveringA}{%
  \mathrel{-\mkern-4mu}<%
}
\newcommand{\coveringB}{\mathrel{\text{$\vcenter{\hbox{\pictcoveringB}}$}}}

\newcommand{\pictcoveringB}{%
  \begin{picture}(1em,.5em)
  \roundcap
  \put(0,.25em){\line(1,0){.6em}}
  \put(.6em,.25em){\line(3,1){.4em}}
  \put(.6em,.25em){\line(3,-1){.4em}}
  \end{picture}%
}
%% Covering

%% Powerset
\newcommand{\powerset}{\raisebox{.15\baselineskip}{\Large\ensuremath{\wp}}}
%% Powerset
\newcommand{\nothing}{\varnothing}

\newtheorem{theorem}{Theorem}

\author{Mat\'u\v{s} Behun}
\title{Lattice theory notes}

\begin{document}

\tableofcontents

\section{Úvod}

%Pri procese rozširovania matematickej teórie vytvárame tvrdenia generalizujúce jej princípy.
%Ak chceme aby náša teória bola správna, všetky jej tvrdenia musia byť logicky odvodené z postulátov alebo tvrdení z nich odvodených.
%Potvrdenie správnosti tvrdenia, vyslovením predpokladu, axiómu alebo napísaním formule ktorú dostaneme aplikáciou dedukčného pravidla na niektoré v postupnosti predchádzajúce formule nazývame dôkazom.

%% Je to pravda?
%Z kvalitatívneho hľadiska pri vyslovení dôkazu uvažujeme o všeobecnosti dôkazu a správnosti aplikácie dedukčného pravidla.
%% Je to vhodný príklad(chcel som uviest priklad na prvy pohlad spravneho tvrdenia ktore bolo dokazane ako nespravne ? Ako správne citovať? Nápad som mal odtiaľto https://en.wikipedia.org/wiki/List_of_disproved_mathematical_ideas
%O nutnosti korektného dokazovania tvrdení hovorí napríklad tvrdenie z teórie čísel o hornom ohraničení počtu prvočísel logaritmickým integrálom.

%\begin{equation}
    %\pi(x) \leq \int_{0}^{x} \frac{1}{ \ln{t} } dt
%\end{equation}

%Tvrdenie bolo považované za správne Bernhardom Riemannom a evidencia to taktiež naznačovala.
%Neskôr sa ukázalo že tvrdenie nie je správne pri čísle pod hodnotou $10^{317}$.
%% Doplniť rok
%Veta o 4 farbách ktorá bola vyslovená v roku 1852 Francisom Guthrie ktorá hovorí, že každá rovinná mapa je zafarbiteľná 4 farbami.
%% 18-faces
%Táto veta bola nesprávne dokázaná v roku Kempom (1879) and Taitom (1880). Kempov dôkaz bol vyvrátený o 10 rokov mapov s 18 stenami.
%% Je lepšie skloňovať cudzie mená?
%Pri dôkaze tejto vety bol neskôr v roku 1977 Appelom and Hakenom z časti využitý počítač pre kontrolu špeciálnych diskrétnych prípadov.

\section{Počítačom asistované dokazovanie}

\section{Prirodzená intuionistická logika}

\subsection{Formalizovanie dôkazu}

Dôkaz z teórie usporiadania. Tak ako je Program = Proof

Otázka ohľadom konzistentnosti dôkazu.

\subsection{Prirodzená dedukcia}

% Formula vyrokoveho poctu
\begin{theorem}[Výroková premenná, formula]
    Majme spočítateľnú množinu $\mathcal{X}$ výrokových premenných. Množina výrokov
    alebo formúl $\mathcal{A}$ generovanú nasledovnou gramatikou:
    \begin{equation}
        A, B ::= X | A \implies B | A \wedge B | A \vee B | \neg A | \top | \bot
    \end{equation}
    Kde $X \in \mathcal{X}$ reprezentuje výrokovú premennú, a $A, B \in \mathcal{A}$
    výrok.
\end{theorem}

V prípade nasledovného výroku je precedencia $\neg$ vyššia ako $\vee$ alebo $\wedge$
a tá je vyššia ako $\implies$. Binárne operátory sú asociatívne z prava.

\begin{align*}
    \neg A \wedge B \wedge C &\implies A \vee B \\
    (\neg A \wedge (B \wedge C)) &\implies (A \vee B) \\
\end{align*}

\begin{theorem}
    Kontextom(systém predpokladov) rozuemieme zoznam výrokov značených
    \begin{equation}
        \Gamma = P_{1}, \dots , P_{n}
    \end{equation}
    Dedukciou nazývame dvojicu pozostávajúcu z kontextu a výroku.
    \begin{equation}
        \Gamma \vdash A
    \end{equation}
\end{theorem}

Výraz čítame ako $A$ je možné dokázať zo systému predpokladov $\Gamma$.

\begin{theorem}
    Dedukčné pravidlo pozostáva z množiny dedukcií $\Gamma_{i}$ ktoré nazývame
    prepokladom. Dolnú časť dedukčného pravidla $\Gamma$ nazývame záverom.
    \begin{equation}
        \begin{prooftree}
            \hypo{\Gamma_{1} \vdash A_{1}}
            \hypo{\dots}
            \hypo{\Gamma_{n} \vdash A_{n}}
            \infer3[]{\Gamma \vdash A}
        \end{prooftree}
    \end{equation}
\end{theorem}
Pravidlá prirodzenej intuicionistickej logiky:
\begin{center}
    \begin{prooftree}
        \infer0[(ax)]{\Gamma,A,\Gamma' \vdash: A}
    \end{prooftree}
\end{center}
\vskip 0.2in
\begin{minipage}[t]{0.48\textwidth}
    \begin{prooftree}
        \hypo{\Gamma \vdash A \implies B}
        \hypo{\Gamma \vdash A}
        \infer2[$(\implies_{E})$]{\Gamma : B}
    \end{prooftree}
\end{minipage}
\hfill
\begin{minipage}[t]{0.48\textwidth}
    \begin{prooftree}
        \hypo{\Gamma, A \vdash B}
        \infer1[$\implies_{I}$]{\Gamma : B}
    \end{prooftree}
\end{minipage}
\vskip 0.2in
\begin{minipage}[t]{0.48\textwidth}
    \begin{prooftree}
        \hypo{\Gamma, A \vdash B}
        \infer1[$(\wedge^{l}_{E})$]{\Gamma : A}
    \end{prooftree}
    \begin{prooftree}
        \hypo{\Gamma, A \vdash B}
        \infer1[$(\wedge^{r}_{E})$]{\Gamma : B}
    \end{prooftree}
\end{minipage}
\hfill
\begin{minipage}[t]{0.48\textwidth}
    \begin{prooftree}
        \hypo{\Gamma \vdash A}
        \hypo{\Gamma \vdash B}
        \infer2[$(\wedge_{I})$]{\Gamma \vdash A \wedge B}
    \end{prooftree}
\end{minipage}
\vskip 0.2in
\begin{minipage}[t]{0.48\textwidth}
    \begin{prooftree}
        \hypo{\Gamma \vdash A \vee B}
        \hypo{\Gamma, A \vdash C}
        \hypo{\Gamma, B \vdash C}
        \infer3[$(\vee_{E})$]{\Gamma \vdash C}
    \end{prooftree}
\end{minipage}
\hfill
\begin{minipage}[t]{0.48\textwidth}
    \begin{prooftree}
        \hypo{\Gamma \vdash B}
        \infer1[$(\vee_{I}^{r})$]{\Gamma \vdash A \vee B}
    \end{prooftree}
    \begin{prooftree}
        \hypo{\Gamma \vdash A}
        \infer1[$(\vee_{I}^{l})$]{\Gamma \vdash A \vee B}
    \end{prooftree}
\end{minipage}
\vskip 0.2in
\begin{minipage}[t]{0.48\textwidth}
    \begin{prooftree}
        \hypo{\Gamma \vdash \neg A}
        \hypo{\Gamma \vdash A}
        \infer2[$(\neg_{E})$]{\Gamma \vdash \bot}
    \end{prooftree}
\end{minipage}
\hfill
\begin{minipage}[t]{0.48\textwidth}
    \begin{prooftree}
        \hypo{\Gamma, A \vdash \bot}
        \hypo{\Gamma \vdash A}
        \infer2[$(\neg_{I})$]{\Gamma \vdash \neg A}
    \end{prooftree}
\end{minipage}
\vskip 0.2in
\begin{center}
    \begin{prooftree}
        \hypo{\Gamma \vdash \bot}
        \infer1[$(\bot_{E})$]{\Gamma \vdash A}
    \end{prooftree}
\end{center}

V prípade že tieto pravidlá čítame zhora nadol hovoríme o dedukcii.
Ak čítame pravidlá zdola nahor hovoríme o indukčnom spôsobe.

\begin{theorem}
    Fragmentom intuionistickej logiky nazývame, systém ktorý dostaneme ak ho obmedzíme
        len na niektoré z predchádzajúcich pravidiel.
\end{theorem}

\begin{theorem}
    Implikačným fragmentom intuionistickej logiky dostaneme v prípade ak formuly
        budú tvorené gramatikou
    \begin{equation}
        A,B ::= X | A \implies B
    \end{equation}
    a pravidlami (ax), ($\implies_{E}$), ($\implies_{I}$)
\end{theorem}

% TODO spravit priklad
% (𝐴∧𝐵)→((𝐴→𝐶)→¬(𝐵→¬𝐶))

V prípade že chceme aby výrokove formuly korenšpondovali s typmi ktoré su prezentované neskôr.
Ich booleova reprezentácia s hodnotami $1,0$ je nahradená otázkou existencie prvkov v množine.
V prípade implikácie o existencii funkcie v množine.
Funkcie v programoch ale môžu mať pri rovnakých vstupoch a výstupoch mať rôznu výpočtovú zložitosť.
Dôvod prečo by sme sa mali pozerať na dôkazy(podľa publikácie Gir11) v troch rovinách.

\begin{itemize}
    \item 1. Booleovský - tvrdenia sú booleovské hodnoty, zaujímame sa o dokázateľnosť tvrdenia
    \item 2. Existenčný - tvrdenia sú množiny, aké funkcie môžu byť
    \item 3. Úmyselný/Zámerový(Intentional) - zaujímame sa o zložitosť vytvoreného dôkazu a ako sa zjednoduší cez (cut eliminitation)
\end{itemize}

\subsection{Intuicionizmus}

Jedným zo smerov matematickej filozofie týkajucej sa rozvoja teórie je konštruktivizmus.
Konštruktivizmus hovorí o potrebe nájsť alebo zostrojiť matematický objekt k tomu
    aby bola dokázaná jeho existencia.
Jeden z motivačných príkladov takéhoto prístupu je možnosť dokázania pravdivosti
výroku $p \vee \neg p$ cez dôkaz sporom $\neg p$ ktorý nehovorí ako zostrojiť objekt
$p$ len o jeho existencii.
Tento smer tvorí viacero "škôl" okrem iných finitizmus, predikativizmus, intuicionizmus.
Intuitionizmus je teda konštruktívny prístup k matematike v duchu
    Brouwera(1881-1966) a Heytinga(1898-1980).
Filozofickým základom tochto prístupu princíp že matematika je výtvorom mentálnej
činnosti a nepozostáva z výsledkov  formálnej manipulácie symbolov ktoré sú iba
sekundárne.
Jedným z princípov intuicionizmus je odmietnutie tvrdenia postulátu klasickej
logiky a to zákona vylúčenia tretieho.

\begin{equation}
    p \vee \neg p
\end{equation}

Dôvodom je z konštruktívneho pohľadu nezmyselnosť uvažovania nad pravdivosťou
    výroku nezávisle od uvažovaného tvrdenia.
Výrok je teda pravdivý ak existuje dôkaz o jeho pravidovsti a nepravdivé
    ak existuje dôkaz ktorý vedie k sporu.

\begin{itemize}
    \item konjukcii $ p \wedge q $ ako o výroku hovoriacom o existencii dôkazov $p$ a zároveň $q$,
    \item disjunkcii $ p \wedge q $ ako existencii konštrukcii dôkazu jedného z výrokov $p, q$,
    \item $ p \implies q $ je metóda(funkcia) transformácie každej konštrukcie $p$ k dôkazu $q$,
    \item neexistencie dôkazu nepravdivého tvrdenia, iba dôkazu ktorý vedie k sporu $p \implies \bot$
    \item konštrukcia $\neg p$ je metóda ktorá vytvorí každú konštrukciu $p$ na neexistujúci objekt
\end{itemize}

konjukcii $ A \wedge B $ ako $ A \times B $
$ A \vee B $ ako $ A \sqcup B $ disjunktne zjednotenie
$ \neg A = A \implies \perp $ existencie kontrapríkladu

\section{Lambda kalkulus}

\begin{theorem}
    Majme nekonečnú množinu $ \mathcal{X}={x,y,z,\dots}$ ktorých elementy nazývame premenné.
Množinu $\Lambda$ tvorenú $\lambda$-termínmy je potom generovaná nasledovnou gramatikou:
    \begin{equation}
        t, u ::= x | t u | \lambda x.t
    \end{equation}
\end{theorem}
\noindent Význam jednotlivých termínov je
\begin{align*}
     x          & \textrm{ - je premennou }\\
     t u        & \textrm{ - je aplikáciou termínu $t$ s argumentom $u$ }\\
    \lambda x.t & \textrm{ - je abstrakciou $t$ nad $x$ }
\end{align*}
Príklady lambda termínov:

\begin{align*}
    & t x \\
    & (\lambda y . \lambda x . t y )) \\
    & (\lambda y.y x) (\lambda x . x) \\
    & t u v = ( t u ) v
\end{align*}

Aplikácia $\lambda$-termínov je implicitne aplikovaná zľava.

Pri výraze
\begin{equation}
    \lambda x . t x = \lambda x . (t x)
\end{equation}
je precedencia aplikácie vyššia ako abstrakcia.

A abstrakciu s troma argumentmi je možné prepísať do troch po sebe nasledujúcich.
\begin{equation}
    \lambda x y z . t = \lambda x . \lambda y . \lambda z . t
\end{equation}

\begin{theorem}
    Premenná x sa vo výraze
    \begin{equation}
        \lambda x . t
    \end{equation}
    abstrakciou viaže na termín $t$. O premennej $x$ hovoríme že je viazaná.
    O premenných ktoré nie sú viazané sú voľné.
    \begin{align*}
        VP(x) &= {x} \\
        VP(\lambda x.t) &= VP(t)  \setminus \{x\} \\
        VP(t v) &= VP(t) \cup VP(v)
    \end{align*}
\end{theorem}

\begin{theorem}
    Premenovaním nazývame nahradenie voľných premenných v termíne.
    \begin{equation}
        t \{ y / x \}
    \end{equation}
\end{theorem}
V termíne $t$ je premenovaná premenná $x$ za $y$.

% TODO pridaj priklady
\subsection{$\alpha$-ekvivalencia}
\begin{theorem}
    O výrazov hovoríme že sú alfa-ekvivalentné ak sa výrazy rovnajú až na premenovanie.
\end{theorem}

\begin{theorem}
    O substutícii hovoríme pri nahradení jednej premenej druhou.
    \begin{equation}
        t [ y / x ]
    \end{equation}
\end{theorem}

Nahradenie je silnejšie a vieme nahradiť aj premmenné viazanné abstrakciou.

\subsection{$\beta$-ekvivalencia}

\begin{minipage}[t]{0.48\textwidth}
    \begin{prooftree}
        \infer0[($\beta_{s}$)]{(\lambda x.t)u \to_{\beta} t [ u / x ]}
    \end{prooftree}
\end{minipage}
\hfill
\begin{minipage}[t]{0.48\textwidth}
    \begin{prooftree}
        \hypo{t \to_{\beta} t'}
        \infer1[($\beta_{\lambda}$)]{(\lambda x.t)u \rightarrow_{\beta} t [ u / x ]}
    \end{prooftree}
\end{minipage}
\vskip 0.2in
\begin{minipage}[t]{0.48\textwidth}
    \begin{prooftree}
        \hypo{t \to_{\beta} t'}
        \infer1[($\beta_{l}$)]{t u \rightarrow_{\beta} t' u}
    \end{prooftree}
\end{minipage}
\hfill
\begin{minipage}[t]{0.48\textwidth}
    \begin{prooftree}
        \hypo{u \to_{\beta} u'}
        \infer1[($\beta_{r}$)]{t u \rightarrow_{\beta} t u'}
    \end{prooftree}
\end{minipage}
\vskip 0.2in

% TODO vymysliet iny strom, tento je prevzaty
\begin{equation}
    \begin{prooftree}
        \infer0[($\beta_{s}$)]{(\lambda y.y)x \to_{\beta} x}
        \infer1[($\beta_{l}$)]{(\lambda y.y)xz \to_{\beta} xz}
        \infer1[($\beta_{\alpha}$)]{\lambda x.(\lambda y.y)xz \to_{\beta} \lambda x . xz}
    \end{prooftree}
\end{equation}

\begin{theorem}
    Definujme rekurziu volania funkcie nasledovne
    \begin{align}
        f^{0}x &= x \\
        f^{n}x &= f(f^{n-1}x) \\
    \end{align}
    Potom Churchove číslo $c_{n}$ je $\lambda$-termín
    \begin{equation}
        c_{n} = \lambda s . \lambda z . s^{n} (z)
    \end{equation}
\end{theorem}

Prirodzené čísla je potom definovať 
\begin{align*}
    0 &= \lambda f x . x \\
    1 &= \lambda f x . f x \\
    1 &= \lambda f x . f (f x) \\
    2 &= \lambda f x . f ( f (f x))
\end{align*}

\begin{align*}
    succ(n) &=           (\lambda n f x .  f( n f x ))(\lambda f x . f^{n} x) \\
           &\to_{\beta} \lambda f x . f (( \lambda f x . f^{n} x ) f x)      \\
           &\to_{\beta} \lambda f x . f (( \lambda x . f^{n} x) x)           \\
           &\to_{\beta} \lambda f x . f (f^{n} x)                            \\
           &=           \lambda f x . f^{n+1} x                              \\
           &= n + 1
\end{align*}

Operáciu sčítania je potom možné definovať vykonať
\begin{theorem}
    $f_{+} = \lambda x. \lambda y. \lambda s. \lambda z. x s (y s z)$
\end{theorem}

Podobným spôsobom môžeme vytvoriť 
\begin{theorem}
    \begin{align*}
        True &= \lambda x y . x \\
        False &= \lambda x y . y
    \end{align*}
\end{theorem}

\begin{align*}
    if = \lambda b x y . b x y
\end{align*}

\begin{align*}
    if \textrm{ True } t u = (\lambda bxy.bxy)(\lambda xy.x) tu & \to_{\beta} (\lambda xy.(\lambda xy.x)xy)tu \\
                                                     & \to_{\beta} (\lambda y.( \lambda xy.x)ty)u \\
                                                     & \to_{\beta} (\lambda xy.x)tu \\
                                                     & \to_{\beta} (\lambda y.t)u \\
                                                     & \to_{\beta} t
\end{align*}

\begin{theorem}
    Jednoduchý $\lambda$ kalkulus je ekvivalentný výpočtovej sile turingovho stroja.
    Bez dôkazu
\end{theorem}

\section{Typovo jednoduchý $\lambda$-calculus}

Typový lambda calculus je rozšírením jednoduchého o typy

\begin{theorem}
    Majme množinu $U$ spočítateľnú nekonečnú abecedu obsahujúcu typové premenné.
    Potom množina $\Pi$ obsahuje reťazce jednoduchých typov ktoré su generované
    gramatikov:
    \begin{equation}
        \Pi ::= U | (\Pi \to \Pi)
    \end{equation}
\end{theorem}

\begin{theorem}
    Kontextom rozumieme množinu $C$ tvoriacu 
    \begin{equation}
        { x_{1} : \tau_{1}, \dots, x_{n} : \tau_{n} }
    \end{equation}
    kde $\tau_{1}, \dots, \tau_{n} \in \Pi$ a $x_{1}, \dots , x_{n} \in$
    Koobor kontextu je množina obsahujúca
    \begin{equation}
        domain(\Gamma) = { x_{1}, \dots, x_{n} }
    \end{equation}
    Oboor kontextu je množina obsahujúca
    \begin{equation}
        range( \Gamma ) = { \tau \in \Pi  | (x : \tau ) \in \Gamma }
    \end{equation}
\end{theorem}

\noindent Príklady generované gramatikou
\begin{itemize}
    \item $\vdash \lambda x.x : \sigma \to \sigma$
    \item $\vdash \lambda x. \lambda y.x : \sigma \to \tau \to \sigma$
    \item $\vdash \lambda x. \lambda y. \lambda z.x z (y z): (\sigma \to \tau \to \rho) \to (\rho \to \tau) \to \sigma \to \rho$
\end{itemize}

\begin{theorem}
    Postupnosť je trojica značená
    \begin{equation}
        \Gamma \vdash t : A
    \end{equation}
tvorená kontextom $\Gamma$, $\lambda$-termínom $t$ a typom $A$.
\end{theorem}

Termín $t$ je typu $A$ ak v kontexte $\Gamma$ ak je postupnosť derivovateľná pomocou pravidiel:
\begin{itemize}
    \item ax: v kontexte $x$ je typu $A$
    \item $\overset{I}{\rightarrow}$: ak je $x$ typu $A$, $t$ je typu B, potom
        funkcia $\lambda x.t$ ktorá asociuje $x$ $t$ je typu $A \to B$
    \item $\overset{E}{\rightarrow}$: daná je funkcia $t$ je typu $A \to B$
        a argument $u$ je typu $A$, vysledok aplikácia $t u$ je typu $B$
\end{itemize}

\begin{center}
    \begin{prooftree}
        \infer0[ax]{\Gamma \vdash x : \Gamma(x)}
    \end{prooftree}
\end{center}
\vskip 0.2in
\begin{minipage}[t]{0.48\textwidth}
    \begin{prooftree}
        \hypo{\Gamma , x : A \vdash t : B }
        \infer1[$\overset{I}{\rightarrow}$]{\Gamma \lambda x^{A}.t : A \to B}
    \end{prooftree}
\end{minipage}
\hfill
\begin{minipage}[t]{0.48\textwidth}
    \begin{prooftree}
        \hypo{\Gamma \vdash t : A \to B }
        \hypo{\Gamma \vdash u : A }
        \infer2[$\overset{E}{\rightarrow}$]{\Gamma \vdash t u : B}
    \end{prooftree}
\end{minipage}

\section{Curry-Howardov izomorfizmus}

\begin{center}
    \begin{tabular}{ c c }
        Intuinistická logika & Typovo jednoduchý $\lambda$ kalkulus \\
        \hline
        termín                  & dôkaz \\
        typová premenná         & propozičná premenná \\
    \end{tabular}
\end{center}

\begin{theorem}{Curry-Howard isomorphism}
    \begin{itemize}
        \item If $\Gamma \vdash M : \varphi \textrm{ potom } |\Gamma|  \vdash \varphi.$
        \item If $\Gamma \vdash \varphi \textrm{ potom existuje } M \in \Lambda_{\Pi}
            \textrm{ také že } \Delta \vdash M : \varphi, \textrm{ kde }
            \Delta = { ( x_{\varphi} : \varphi) | \varphi \in \Gamma }$
    \end{itemize}
\end{theorem}


\section{Lean-theorem-prover}
\subsection{Constracting proof}
\subsection{Forward proofs}
    \subsubsection{assume}
    \subsubsection{calc}
    \subsubsection{fix}
    \subsubsection{have}
    \subsubsection{let}
    \subsubsection{show}
\subsection{Backward proofs}
    \subsubsection{cc}
    \subsubsection{clear}
    \subsubsection{exact}
    \subsubsection{induction}
    \subsubsection{intro}
    \subsubsection{refl}
    \subsubsection{refl}
\subsubsection{Inductive types}

\end{document}

% sposob ukladania kodu
\begin{verbatim}
    structure point :=
      ( x : nat )
      ( y : nat )

    /-- alternative notation -/
    structure point_alternative :=
      mk :: (x : nat) (Y : nat)

    def p1 : point :=
    {
      x   := 10,
      y   := 20,
    }

    /- same point, different notation, same notation for ordered seti -/
    def p2 : point := $\langle 10, 20 \rangle$

    /- instance only one part of structure, rest implicitly from other instance
    def p3 : point := {
        x := 20,
        ..p
    }
\end{verbatim}

\subsubsection{Type classes}

\begin{theorem} \emph{Veta o izomorfizme modulárnych zväzov}
Nech L je modulárnym zväzom a $a, b \in L$. Potom
    \begin{equation}
        \varphi_{b}: x \mapsto x \wedge b, x \in [a, a \vee b],
    \end{equation}
Je izomorfizmom medzi intervalmi $[a, a \vee b]$ a $[ a \wedge b, b]$.
Inverzným izomorfizmom je
    \begin{equation}
        \psi_{a}: y \mapsto x \vee a, y \in [a \wedge b, b].
    \end{equation}
\end{theorem}
\emph{Dôkaz}.
Stačí ukázať že $\varphi_{b}\psi_{a}(y) = y$ pre všetky $x \in [a, a \vee b]$.
Z duality vyplýva že $\varphi_{b}\psi_{a}(y) = y$ pre všetky
    $y \in [a \wedge b, b]$,
Majme $x \in [a, a \vee b]$. Potom
    $\psi_{a}\varphi_{b} = ( x \wedge b ) \vee a$ nerovnosť $a \leq x$ platí
    potom aj modularita
    \begin{equation}
        \varphi_{a}\psi_{b}(x) =
        ( x \wedge b ) \vee a =
        x \wedge ( b \vee a) =
        x
    \end{equation}
    pretože
    \[
        \pushQED{\qed}
        x \leq a \vee b. \qedhere
        \popQED
    \]
