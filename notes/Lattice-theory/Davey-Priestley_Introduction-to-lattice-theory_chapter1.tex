\documentclass[a4paper,12pt,oneside]{report}%pridat twoside, do [] pre obojstrannu tlac
\pagestyle{headings}
\usepackage[top=2.5cm, bottom=2.5cm, left=3.5cm, right=2cm]{geometry} %odporucane okraje
\linespread{1.50}

%% Generally used
\usepackage{amsthm}
\usepackage{amsmath}
\usepackage{amssymb}
\usepackage{mathtools}
%% Generally used

%% Covering
\usepackage{pict2e,picture}

\newcommand{\coveringA}{%
  \mathrel{-\mkern-4mu}<%
}
\newcommand{\coveringB}{\mathrel{\text{$\vcenter{\hbox{\pictcoveringB}}$}}}

\newcommand{\pictcoveringB}{%
  \begin{picture}(1em,.5em)
  \roundcap
  \put(0,.25em){\line(1,0){.6em}}
  \put(.6em,.25em){\line(3,1){.4em}}
  \put(.6em,.25em){\line(3,-1){.4em}}
  \end{picture}%
}
%% Covering

%% Powerset
\newcommand{\powerset}{\raisebox{.15\baselineskip}{\Large\ensuremath{\wp}}}
%% Powerset

\newtheorem{theorem}{Theorem}

\author{Mat\'u\v{s} Behun}
\title{Lattice theory notes}

\begin{document}
\section{Chapter 1}

\begin{theorem}
Let $P$ be a set. An order (or \textbf{partial order}) on $P$ is a binary
    relation $\leq$ on $P$ such that, for all $x, y, z \in P$,
\begin{enumerate}
    \item $x \leq x$,
    \item $x \leq y$ and $y \leq x$ imply $x = y$,
    \item $x \leq y$ and $y \leq z$ imply $x \leq z$.
\end{enumerate}
\end{theorem}

\begin{theorem}
Let $P$ be an ordered set. Then $P$ is a \textbf{chain} if, for all
    $x, y \in P$, either $x \leq y$ or $y \leq x$ (that is, if any two elements
    of $P$ are comparable).
    Alternative names for a chain are \textbf{linearly ordered set} and
    \textbf{totally ordered set}.
The ordered set is an \textbf{antichain} if $x \leq y$ in $P$ only if $x ==y$.
\end{theorem}

\begin{theorem}
We say that $P$ and $Q$ are (\textbf{order-})\textbf{isomorphic}, and write
    $P \cong Q$, if there exists a map $\varphi$ from $P$ onto $Q$ such that
    $x \leq y$ in $P$ if and only if $\varphi(x) \leq \varphi(y)$ in $Q$. 
Then $\varphi$ is called \textbf{order-isomorphism}.
\end{theorem}

\begin{theorem}
Let $P$ be an ordered set and let $x,y \in P$. We say $x$ is \textbf{covered by}
    $y$ (or $y$ \textbf{covers} $x$), and write $x \coveringA y$, if $x \le y$
    and $x \leq z \le y$ implies $z = x$.
The latter condition is demanding that there be no element $z$ of $P$ with
    $x \le z \le y$.
\end{theorem}

\begin{theorem}
Let $P$ be a finite ordered set. We can represent $P$ by a configuration of
    circles (representing the elements of $P$) and interconnecting lines
    (indicating the covering relation). The construction goes as follows.
    \begin{enumerate}
        \item To each point $x \in P$, associate a point $p(x)$ of the Euclidean
            plane, depicted by a small circles with centre at $p(x)$.
        \item For each covering pair $x \coveringA y$ in $P$, take a line
            segment $\ell(x, y)$ joining the circle at $p(x)$ to the circle at
            $p(y)$.
        \item Carry out 1 and 2 in such a way that
            \begin{itemize}
                \item if $x \coveringA y$, then $p(x)$ is `lower' than $p(y)$
                    (that is, in standard Cartesian coordinates, has strictly
                    smaller second coordinate),
                \item the circle at $p(z)$ does not intersect the line segment
                    $\ell(x,y)$ if $z \ne x$ and $z \ne y$.
            \end{itemize}
    \end{enumerate}
\end{theorem}

\begin{theorem}
Let $P$ and $Q$ be finite ordered sets and let $\varphi: P \mapsto Q$ be a
    bijective map.
Then the following are equivalent:
    \begin{enumerate}
        \item $\varphi$ is an order-isomorphism;
        \item $x \le y$ in $P$  if and only  if $\varphi(x) \le \varphi(y)$ in
            $Q$
        \item $x \coveringA y$ in $P$ if and only if
            $\varphi(x) \coveringA \varphi(y)$ in $Q$
    \end{enumerate}
\end{theorem}

\begin{theorem}
Given any ordered set $P$ we can form a new orderer set $P^{\partial}$ (the
    \textbf{dual} of $P$) by defining $x \leq y$ to hold in $P^{\partial}$ if
    and only if $y \leq x$ hold in $P$.
For $P$ finite, we obtain a diagram for $P^{\partial}$ simply by
    `turning upside down' a diagram for $P$.
\end{theorem}

\begin{theorem}
Let $P$ be an ordered set. We say $P$ has a bottom element if there exists
    $\bot \in P$ (called \textbf{bottom}) with the property that $\bot \leq x$
    for all $x \in P$.
Dually,  $P$ has a top element if there exists $\top \in P$ such that
    $x \leq \top$ for all $x \in P$.
\end{theorem}

\begin{theorem}
Let $P$ be an ordered set. We say $P$ has a bottom element if there exists
    $\bot \in P$ (called \textbf{bottom}) with the property that $\bot \leq x$
    for all $x \in P$.
Dually,  $P$ has a top element if there exists $\top \in P$ such that
    $x \leq \top$ for all $x \in P$.
\end{theorem}

\begin{theorem}
Given an ordered set $P$ (with or without $\bot$), we form $P_{\bot}$ (called
    $P$ `lifted') as follows. Take an element $\textbf{0} \notin P$ and define
    $\leq$ on $P_{\bot} := P \cup \lbrace\textbf{0}\rbrace$ by
    \begin{center}
        $x \leq y$ if and only if $x = \textbf{0}$ or $x \leq y$ in $P$
    \end{center}
\end{theorem}

\begin{theorem}
Let $P$ be an ordered set and let $Q \subseteq P$. Then $a \in Q$ is a
    \textbf{maximal element} of $Q$ if $a \leq x$ and $x \in Q$ imply $a = x$.
We denote the set of maximal elements of $Q$ by $MaxQ$.If $Q$ (with the order
    inherited from $P$) has a top element, $\top_{Q}$, then 
    $MaxQ = \lbrace \top_{Q} \rbrace$; in this case $\top_{Q}$ is called the
    \textbf{greatest} (or \textbf{maximum}) element of $Q$, and we write
    $\top_{Q} = maxQ$. 
A \textbf{minimal} elment of $Q \subseteq P$ and $minQ$, the \textbf{least} (or 
    \textbf{minimum}) element of $Q$ (when these exist) are defined dually, that 
    is by reversing the order.
\end{theorem}

\begin{theorem}
Suppose that $P$ and $Q$ are (disjoint) ordered sets. The
    \textbf{disjoint union} $P \cup Q$ of $P$ and $Q$ is the ordered set formed
    by defining $x \leq y$ in $P \cup Q$ if and only if either $x,y \in P$ and
    $x \leq y$ in $P$ or $x,y \in Q$ and $x \leq y$ in $Q$.
A diagram for $P \cup Q$ is formed by placing side by side diagrams for $P$ and 
    $Q$.
\end{theorem}

\begin{theorem}
Let $P$ and $Q$ be (disjoint) ordered sets. \textbf{The linear sum}
    $P \oplus Q$ is defined by taking the following order relation on
    $P \cup Q: x \leq y$ if and only if
    \begin{align*}
                         &x,y \in P \textnormal{ and } x \leq y \textnormal{ in } P, \\
        \textnormal{or } &x,y \in P \textnormal{ and } x \leq y \textnormal{ in } Q, \\
        \textnormal{or } &x   \in P \textnormal{ and } y \in Q
    \end{align*}
A diagram for $P \oplus Q$ (when $P$ and $Q$ are finite) is obtained by placing
    a diagram for $P$ directly below a diagram for $Q$ and then adding a line
    segment from each \emph{each} maximal element of $P$ to \emph{each} minimal
    element of $Q$.
\end{theorem}

\begin{theorem}
    Let $P_{1}, \ldots, P_{n}$ be ordered sets. The Cartesian product
    $P_{1} \times \ldots \times P_{n}$ can be made into an ordered set by imposing
    the coodinatewise order by
    \begin{align*}
        (x_{1}, \ldots, x_{n}) \leq (y_{1}, \ldots, y_{n})
            \iff (\forall i) x_{i} \leq y_{i} \textnormal{ in } P_{i}.
    \end{align*}
\end{theorem}

\begin{theorem}
Let $X = \lbrace 1,2, \ldots ,n \rbrace$ and define
    $\varphi: \powerset(X) \rightarrow \powerset{2}^{n}$ by
    $\varphi(A) = ( \epsilon_{1}, \ldots, \epsilon_{n} )$ where
    \[
        \epsilon_{i} =
        \begin{cases}
            1 \textnormal{ if } i \in A, \\
            0 \textnormal{ if } i \notin A. \\
        \end{cases}
    \]
Then $\varphi$ is an order-isomorphism.
\end{theorem}

\begin{theorem}
Let $P$ be an ordered set and $Q \subseteq P$.
    \begin{itemize}
        \item $Q$ is a \textbf{down-set} (alternative term include
            \textbf{decreasing set} and \textbf{order ideal}) if, whenever
            $x \in Q$, $y \in P$ and $y \leq x$, we have $y \in Q$.
        \item Dually, $Q$ is an \textbf{up-set} (alternative terms are
            \textbf{increasing set} and \textbf{order filter}) if, whenever
            $x \in Q$, $y \in P$ and $y \geq x$, we have $y \in Q$.
    \end{itemize}
\end{theorem}

\begin{theorem}
    \begin{center}
    $\downarrow Q := \lbrace y \in P | ( \exists x \in Q ) y \leq x \rbrace$ and
    $\uparrow   Q := \lbrace y \in P | ( \exists x \in Q ) y \geq x \rbrace$,
    \end{center}

    \begin{center}
    $\downarrow x := \lbrace y \in P | y \leq x \rbrace$ and
    $\uparrow   x := \lbrace y \in P | y \geq x \rbrace$.
    \end{center}
\end{theorem}

\begin{theorem}
The family of all down-sets of $P$ is denoted by $\mathcal{O}(P)$. It is itself
    an ordered set, under the inclusion order.
\end{theorem}

\begin{theorem}
Let $P$ be an ordered set and $x,y \in P$. Then the following are equivalent:
    \begin{enumerate}
        \item $x \leq y$;
        \item $\downarrow x \subseteq \downarrow y$;
        \item $(\forall Q \in \mathcal{O}(P)) y \in Q \implies x \in Q$.
    \end{enumerate}
\end{theorem}

\begin{theorem}
    \begin{align*}
        \mathcal{O}(P)^{\partial} = \mathcal{O}(P^{\partial})
    \end{align*}
\end{theorem}

\begin{theorem}
Let $P$ be an ordered set. Then
    \begin{enumerate}
        \item $\mathcal{O}(P \oplus \textbf{1}) \cong \mathcal{O}(P) \oplus \textbf 1$
            and $\mathcal{O}(\textbf{1} \oplus P) \cong \textbf{1} \oplus \mathcal{O}(P)$;
        \item $\mathcal{O}(P_{1} \cup P_{2}) \cong \mathcal{O}(P_1) \times \mathcal{O}(P_{2})$.
    \end{enumerate}
\end{theorem}

\begin{theorem}
Let $P$ and $Q$ be ordered sets. A map $\varphi: P \rightarrow Q$ is said to be
    \begin{enumerate}
        \item \textbf{order-preserving} (or alternatively, \textbf{monotone}) if
            $x \leq y$ in $P$ implies $\varphi(x) \leq \varphi(y) in Q$;
        \item an \textbf{order-embedding} (and we write 
            $\varphi: P \rightarrow Q$) if $x \leq y$ in $P$ if and only if
            $\varphi(x) \leq \varphi(y)$ in $Q$;
        \item an \textbf{order-isomorphism} if it is an order-embedding which
            maps $P$ onto $Q$.
    \end{enumerate}
\end{theorem}
\end{document}
