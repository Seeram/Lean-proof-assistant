\documentclass[a4paper,12pt,oneside]{report}%pridat twoside, do [] pre obojstrannu tlac
\pagestyle{headings}
\usepackage[top=2.5cm, bottom=2.5cm, left=3.5cm, right=2cm]{geometry} %odporucane okraje
\linespread{1.50}

%% Generally used
\usepackage{amsthm}
\usepackage{amsmath}
\usepackage{amssymb}
\usepackage{mathtools}
%% Generally used

%% Covering
\usepackage{pict2e,picture}

\newcommand{\coveringA}{%
  \mathrel{-\mkern-4mu}<%
}
\newcommand{\coveringB}{\mathrel{\text{$\vcenter{\hbox{\pictcoveringB}}$}}}

\newcommand{\pictcoveringB}{%
  \begin{picture}(1em,.5em)
  \roundcap
  \put(0,.25em){\line(1,0){.6em}}
  \put(.6em,.25em){\line(3,1){.4em}}
  \put(.6em,.25em){\line(3,-1){.4em}}
  \end{picture}%
}
%% Covering

%% Powerset
\newcommand{\powerset}{\raisebox{.15\baselineskip}{\Large\ensuremath{\wp}}}
%% Powerset

\newtheorem{theorem}{Theorem}

\author{Mat\'u\v{s} Behun}
\title{Lattice theory notes}

\begin{document}
\begin{theorem}
Let $L$ be a modular lattice and let $a,b \in L$. Then
    \begin{align*}
        \varphi_{b}: x \mapsto x \wedge b, x \in [a, a \vee b],
    \end{align*}
is an isomorphism between the intervals $[a, a \vee b]$ and $[a \wedge b, b]$.
The inverse isomorphism is
    \begin{align*}
        \psi_{a}: y \mapsto x \vee a, y \in [a \vee b, b].
    \end{align*}
\emph{Proof.} It is sufficient to show that $\psi_{a}\varphi{b}(x)$ = x for
all $x \in [a, a \vee b]$.
Indeed, if this is true, then by duality, $\varphi_{b}\psi_{a}(y) = y$, for all 
    $y \in [a \wedge b, b]$, is also true.
The isotone maps $\varphi_{b}$ and $\psi_{a}$,  thus compose into the identity maps,
hence they are isomorphisms, as claimed. \\
So let $x \in [a, a \vee b]$.
Then $\psi_{a}\varphi{b}(x) = (x \wedge b) \vee a$.
Since $x \in [a, a \vee b]$, the inequality $a \leq x$ holds, and so modularity 
    applies:
    \begin{align*}
        \psi_{a}\varphi_{b}(x) = ( x \wedge b) \vee a = x \wedge ( b \ vee a ) = x,
    \end{align*}
because $x \leq a \vee b$. \qedsymbol
\end{theorem}


\end{document}
